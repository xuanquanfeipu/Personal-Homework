\section{系统结构} \label{sys-arch}
\qquad 
在 src 中主要有这样几个文件:

\begin{itemize}
\item {\bf main.cpp} 主要实现了和用户交互的逻辑
\item {\bf vertex.h} 主要实现了基本的节点类型 Vertex
\item {\bf path.h/cpp} 主要实现了基本的路径类型 Path
\item {\bf route.h/cpp} 整个路径规划算法的抽象接口 Route,同时定义了一些辅助函数用于输出等。下面三个是这个抽象类的具体实现:
	\begin{itemize}
	\item {\bf route\_network\_flow.h/cpp} 基于网络流的布线方法具体实现
	\item {\bf route\_rule\_based.h/cpp} 基于规则的布线方法的具体实现
	\item {\bf route\_input\_adapter.h/cpp} 用于读取存储于文件中的路径信息的适配器
	\end{itemize}
\item {\bf visualization.h/cpp} 用于可视化算法输出
\end{itemize}

Route 的具体实现都不对用户可见,利用一个委托函数来创建其实例同时转换为基类的指针。为了减少用户手动删除造成的不必要错误,使用 std::shared\_ptr 来管理内存。具体实现大致如下:

\begin{minted}{c++}
namespace {
    class RouteNetworkFlowImpl : public Route
    {
    public:
        void compute() {
            // implementation
        }
    };
} // unnamed namespace

RoutePtr get_network_flow_algo()
{
    return std::make_shared<RouteNetworkFlowImpl>();
}
\end{minted}

